% A
\newacronym{api}{API}{Application Programming Interface}

% C
\newacronym{core}{CoRE}{Constrained RESTful Environment}
\newglossaryentry{consumer}{
  name={Consumer},
  description={Eine Entität, die Thing Descriptions verstehen und mit Things über diese interagieren kann.} % TODO: Kann mit Consumer im PubSub-Kontext verwechselt werden
  % Mögl. Alternativen: Nutzer, Akteur
}

% H
\newacronym{html}{HTML}{Hypertext Markup Language}
\newacronym{http}{HTTP}{Hypertext Transfer Protocol}

% I
\newglossaryentry{intermediary}{
  name={Intermediary},
  plural={Intermediaries},
  description={Eine Entität zwischen Consumern und Things, die Things erweitern, zusammenstellen oder vertreten kann und eine Thing Description bereitstellt, die die Zugriffe auf Things kapselt. Ein Intermediary könnte daher von außen nicht von einem Thing unterschieden werden.}
}
\newacronym{iot}{IoT}{Internet of Things}
\newacronym{iri}{IRI}{Internationalized Resource Identifier}

% J
\newacronym{json}{JSON}{JavaScript Object Notation}
\newacronym{jsonld}{JSON-LD}{\glsxtrshort{json} for Linked Data}

% L
\newacronym{lan}{LAN}{Local Area Network}
\newacronym{lpwan}{LPWAN}{Low Power Wide Area Network}

% P
\newglossaryentry{pubsub}{
  name={PubSub},
  description={Ein Nachrichtenübermittlungsmuster, bei dem Sender von Nachrichten, auch \enquote*{Publisher} genannt, nicht direkt an bestimmte Empfänger, auch \enquote*{Subscriber} genannt, sendern, sondern veröffentlichte Nachrichten in Kategorien eingeteilt werden, während die konkreten \enquote*{Subscriber} unbekannt sind, wenn es überhaupt welche gibt. Auf ähnliche Art und Weise abonnieren \enquote*{Subscriber} bestimmte Kategorien, ohne zu wissen, ob überhaupt \enquote*{Publisher} zu dieser Kategorie beitragen und wenn ja, welche \enquote*{Publisher} das sind.}
}

%
\newacronym{qos}{QoS}{Quality of Service}

% R
\newacronym{rest}{REST}{Representational State Transfer}

% S
\newacronym{sd}{SD}{Service Discovery}
\newacronym{SSE}{SSE}{Server-Sent Events}

% T
\newglossaryentry{thing}{
  name={Thing},
  description={Hardware/Software, die kommunizieren kann, programmierbar ist und entweder Sensoren besitzt oder Elemente in Bewegung versetzen kann.}
}
\newacronym{td}{TD}{Thing Description}
\newacronym[\glslongpluralkey={Thing Description Directories}]{tdd}{TDD}{Thing Description Directory}

% U
\newacronym{uri}{URI}{Uniform Resource Identifier}
\newacronym{url}{URL}{Uniform Resource Locator}

% W
\newacronym{wot}{WoT}{Web of Things}
\newglossaryentry{wt}{
  name={Web Thing},
  description={Digitale Repräsentation eines Things und dessen Fähigkeiten mittels Webtechnologien. (Je nach Implementierung im Speziellen mit einigen ggf. wichtigen Unterschieden, aber das ist allgemeiner Konsens.)}
}
\newacronym{w3c}{W3C}{World Wide Web Consortium}
