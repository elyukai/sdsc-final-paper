\section{Conclusion}\label{sec:conclusion}

We implemented a reference of how to serve \glspl{td} and the variety of search options in a \gls{wot} context. A mobile application was used to show realistic use cases.

The lack of a standard ontology for Smart City contexts in the \gls{wot} specification has been solved by creating a new ontology based on established ontologies. It has been designed for easy extension of missing sensor types and the integration of search-relevant \gls{qos} aspects.

Despite the missing ontology, the proven expressiveness of and widespread availability of required hard- and software for the \gls{wot} suits \gls{sd} in open data contexts, providing a unifying concept for making different data endpoints available to a variety of users.

The usage of \gls{jsonld} throughout the \gls{wot} imposes normalization and verbosity difficulties on both servers and clients, especially if syntactic search methods are used, due to the possible heterogeneity of multiple \glspl{td}.
