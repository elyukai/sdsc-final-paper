\section{Introduction}\label{sec:introduction}

The deployment of \gls{iot} devices and sensors in cities has increased significantly in recent years, and these systems play a crucial role in collecting data that provides various information about the environment.
For example, sensors can measure the current temperature in the city center, or help a driver locate unused parking spots in the city center.
In this paper, we investigate the development of a \gls{wot} solution for the standardization of \gls{iot} systems interfaces used by cities.

The problem is that the interfaces of these sensors are not consistent, making it difficult for developers to access the data.
A European directive \autocite{digital-strategy.ec.europa.eu} requires member states, among other things, to make the data collected by sensors available to the public, but they have the freedom to decide how to do so.
This means that each city can use its own standards, which complicates interoperability. This makes it difficult for developers to create applications that can use data from multiple sources.
Therefore, it is necessary to find a solution that allows data from different sources to be combined and a standard format to be set.

The authors of this paper previously conducted a study titled \citetitle{Survey.Service.Discovery.in.Smart.Cities} \autocite{Survey.Service.Discovery.in.Smart.Cities} in which various solutions were examined.
In this study, the \gls{wot} technology emerged as the best solution. It is used to extend the functionality of \gls{iot} and optimize \gls{sd}.

The goal of this study was to develop an extended \gls{wot} solution that allows cities to combine and make data from different \gls{iot} systems accessible.
Furthermore, it will be demonstrated using a test environment that the solution can be successfully implemented and utilized.

As part of this paper, a backend system was developed to store and manage sensor information.
The backend system uses a format that is based on \gls{wot} and has been extended by us.
It allows for the integration and processing of data from various sensors and devices.
Additionally, a mobile application was created as a frontend system that works with sensor data to test the backend.
This application demonstrates how developers can use data from the backend to create applications that work with sensor data.
In our example, the application can be used to display temperatures at different locations.

We have shown that it is possible to develop a backend that uses this new standard and that it is possible to create applications that can work with this data.
We believe that the development of an extended \gls{wot} solution for the standardization of \gls{iot} systems interfaces will greatly benefit cities and developers by providing a unified and easily accessible data source.

Initially, important fundamentals will be explained, in order to subsequently delve into the relevant problem.
Later on, the solution will be presented and the structure of the corresponding test environment will be explained.
Afterwards, we will discuss the work and draw a conclusion.
