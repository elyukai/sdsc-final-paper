\subsection{Semantic \glsentrylong{sd}}\label{sec:semanticsd}

The main aspect of semantic \gls{sd} is to find services based on an ontology. An ontology is a way of describing how different properties are related to each other. They are used to build a shared vocabulary and unambiguous meaning across a subject area.
While many other \glspl{sd} only work with functional properties (e.g. accepted message types, required inputs, \dots), semantic \gls{sd} uses functional and non-functional (e.g. performance, cost, reliability, \dots) properties alike.

How a semantic \gls{sd} is structured is highly dependent on the purpose it was built for and who it was built by. Since there is no standardization in place the discovery process as well as the used ontology can be adjusted to fit the developer's needs. This makes it possible for \citeauthor{Iqbal.12320081252008.SSDuSaS} \autocite{Iqbal.12320081252008.SSDuSaS} to describe services using \emph{SAWSDL} and \emph{SPARQL} while \citeauthor{BenMokhtar.2006.ESSDiPCE} \autocite{BenMokhtar.2006.ESSDiPCE} use \emph{Amigo-S} and \emph{S-Ariadne} in their work. There are also a lot of different approaches on how to select suitable services for users \autocite{Majithia.2004.Rbssd} \autocite{Jia.2017} \autocite{Zhao.2017}.
