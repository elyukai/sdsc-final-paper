\subsection{Ontology}\label{sec:ontology}

To overcome this problem, we propose a new ontology that is based on already established ontologies and aligns with the \citetitle{w3c.wot.tdo.20230111} for specific use in \glspl{td} and, at the same time, an easy way to extend the ontology if new application areas are needed.

The core element of the ontology is the \lstinline|Sensor| class, which is built on top of the \lstinline|Sensor| class from \citetitle{etsi.saref.core} and its extensions from the \citetitle{etsi.saref.envi}. Also inheriting properties from the \citetitle{w3c.geo}, the \lstinline|Sensor| class provides important \gls{qos} features such as the location, measurement frequency and data transmission period. The class is intended to be sub-classed for each sensor type to have a common set of important values available to retrieve independent of the sensor type.

Each sensor type consists of a set of different classes and named individuals, as seen in \autoref{fig:ontology-parts-per-sensor-type}. This includes the parts which are the same across all sensor types, which makes it easy to extend the ontology with new sensor types.

\begin{figure}[H]
  \begin{tikzpicture}[
    node distance=30mm and 15mm,
    on grid,
    text height=1.5ex, % size including ascenders
    text depth=.1ex,   % descenders size
    class/.style={
      draw=ACMPurple,
      inner xsep=6pt,
      inner ysep=6pt,
      rectangle,
      draw,
      line width=1pt,
    },
    namedindividual/.style={
      draw=ACMOrange,
      inner xsep=6pt,
      inner ysep=6pt,
      rounded rectangle,
      draw,
      line width=1pt,
    },
    edgedesc/.style={
      font=\ttfamily\scriptsize,
    },
  ]
    \node (s) [class] {Sensor};
    \node (pa) [class,below right=of s,xshift=10mm] {Property Affordance}
      edge [<-] node [edgedesc,above,sloped,xshift=1mm] {td:hasPropertyAffordance} (s);
    \node (p) [class,below left=of pa] {Property}
      edge [<-] node [edgedesc,below,sloped,yshift=2pt] {saref:measuresProperty} (s)
      edge [<-] node [edgedesc,above,sloped]            {ssn:forProperty}        (pa);
    \node (pu) [class,below=of p,xshift=-5mm,yshift=10mm] {Property Unit}
      edge [<-] node [edgedesc,right] {om:commonlyHasUnit} (p);
    \node (u1) [namedindividual,below left=of pu,yshift=10mm] {Unit 1}
      edge [->] node [edgedesc,above,sloped,xshift=-1mm] {rdf:type} (pu);
    \node (u2) [namedindividual,below=of pu,yshift=10mm] {Unit 2}
      edge [->] node [edgedesc,above,sloped] {rdf:type} (pu);
    \node (u2) [namedindividual,below right=of pu,yshift=10mm] {…}
      edge [->] node [edgedesc,above,sloped,xshift=1mm] {rdf:type} (pu);
    \node (m) [class,below left=of s,xshift=-10mm] {Measurement}
      edge [->] node [edgedesc,above,sloped,xshift=-1mm]  {saref:measurementMadeBy} (s)
      edge [->] node [edgedesc,above,sloped,xshift=1.5mm] {saref:relatesToProperty} (p)
      edge [->] node [edgedesc,below,sloped,yshift=1pt]   {saref:isMeasuredIn}      (pu);
  \end{tikzpicture}
  \caption{Ontology parts for each sensor type}\label{fig:ontology-parts-per-sensor-type}
\end{figure}

The class that subclasses \lstinline|Sensor| represents the root of the \gls{td}. The \lstinline|Measurement| is not a direct part of the \gls{td}, which is defined by the other parts mentioned in \autoref{fig:ontology-parts-per-sensor-type}, but serves as the definition of the format used to serve the data of a single measurement. This ensures a unified data format for all sensors of a type. The \lstinline|Property| mainly serves as a description of the actual property to measure and to bind the property's unit classes. The \lstinline|Property Unit| is only used to provide a common type for all named individuals that represent each possible unit of the property.

We applied this principle to multiple measurable values to verify it works for different use cases. Hereafter, we will discuss one of those examples in detail.

Our ontology is defined using Turtle syntax \autocite{w3c.turtle}, since most of the existing ontologies are also defined using this syntax. To fully understand our ontology, we first introduce the used prefixes and their expanded \gls{uri} forms in \autoref{lst:ontology-prefixes}.

\begin{lstlisting}[language=turtle,caption={Ontology prefixes},label={lst:ontology-prefixes}]
@prefix : <https://sdsc.fly.dev/ontology#> .
@prefix owl: <http://www.w3.org/2002/07/owl#> .
@prefix rdf: <http://www.w3.org/1999/02/22-rdf-syntax-ns#> .
@prefix xml: <http://www.w3.org/XML/1998/namespace> .
@prefix xsd: <http://www.w3.org/2001/XMLSchema#> .
@prefix rdfs: <http://www.w3.org/2000/01/rdf-schema#> .
@prefix saref: <https://saref.etsi.org/core/> .
@prefix s4envi: <https://saref.etsi.org/saref4envi/> .
@prefix ssn: <http://www.w3.org/ns/ssn/> .
@prefix geo: <http://www.w3.org/2003/01/geo/wgs84_pos#> .
@prefix td: <https://www.w3.org/2019/wot/td#> .
@prefix om: <http://www.ontology-of-units-of-measure.org/resource/om-2/> .
@base <https://sdsc.fly.dev/ontology> .
\end{lstlisting}

The \lstinline|Sensor| class is the core class (\autoref{lst:sensor-class}). It is only defined once and is sub-classed by \lstinline|TemperatureSensor|, which is our example sensor type and on which its measured property and the temperature affordance for the \gls{td} are defined (\autoref{lst:temperaturesensor-class}). The \lstinline|Sensor| class also makes it possible to include \gls{qos} aspects like location, the frequency of measurements and the transmission period.

\begin{lstlisting}[language=turtle,caption={{\ttfamily{}Sensor} class},label={lst:sensor-class}]
###  https://sdsc.fly.dev/ontology#Sensor
:Sensor
  rdf:type owl:Class ;
  rdfs:subClassOf
    geo:SpatialThing ,
    saref:Sensor ,
    [ rdf:type owl:Restriction ;
      owl:onProperty s4envi:hasFrequencyMeasurement ;
      owl:allValuesFrom s4envi:FrequencyMeasurement
    ] ,
    [ rdf:type owl:Restriction ;
      owl:onProperty s4envi:hasTransmissionPeriod ;
      owl:allValuesFrom s4envi:PeriodMeasurement
    ] .
\end{lstlisting}

\begin{lstlisting}[language=turtle,caption={{\ttfamily{}TemperatureSensor} class},label={lst:temperaturesensor-class}]
###  https://sdsc.fly.dev/ontology#TemperatureSensor
:TemperatureSensor
  rdf:type owl:Class ;
  rdfs:subClassOf
    :Sensor ,
    [ rdf:type owl:Restriction ;
      owl:onProperty saref:measuresProperty ;
      owl:someValuesFrom :Temperature
    ] ,
    [ rdf:type owl:Restriction ;
      owl:onProperty td:hasPropertyAffordance ;
      owl:someValuesFrom :TemperatureAffordance
    ] .
\end{lstlisting}

The \lstinline|TemperatureAffordance| class then restricts the permitted values for its property affordance to \lstinline|Temperature| values (\autoref{lst:temperatureaffordance-class}).

\begin{lstlisting}[language=turtle,caption={{\ttfamily{}TemperatureAffordance} class},label={lst:temperatureaffordance-class}]
###  https://sdsc.fly.dev/ontology#TemperatureAffordance
:TemperatureAffordance
  rdf:type owl:Class ;
  rdfs:subClassOf
    [ rdf:type owl:Restriction ;
      owl:onProperty ssn:forProperty ;
      owl:someValuesFrom :Temperature
    ] .
\end{lstlisting}

A \lstinline|Temperature| is defined by a common unit and can be labelled and described (\autoref{lst:temperature-class}). The associated \lstinline|TemperatureUnit| is defined as a class, which each of its actual units as named individuals (\autoref{lst:temperatureunit-class}). Those individuals are imported from another ontology and mapped onto the \lstinline|TemperatureUnit| class.

\begin{lstlisting}[language=turtle,caption={{\ttfamily{}Temperature} class},label={lst:temperature-class}]
###  https://sdsc.fly.dev/ontology#Temperature
:Temperature
  rdf:type owl:Class ;
  rdfs:subClassOf saref:Property ;
  om:commonlyHasUnit :TemperatureUnit ;
  rdfs:comment "Temperature is the extent to which an object is hot."@en ;
  rdfs:label "Temperature"@en .
\end{lstlisting}

\begin{lstlisting}[language=turtle,caption={{\ttfamily{}TemperatureUnit} class and its named individual(s)},label={lst:temperatureunit-class}]
###  https://sdsc.fly.dev/ontology#TemperatureUnit
:TemperatureUnit rdf:type owl:Class ;
  rdfs:subClassOf
    om:Unit ,
    saref:UnitOfMeasure ;
  rdfs:comment "The unit of measure for temperature"@en ;
  rdfs:label "Temperature unit"@en .

###  http://www.ontology-of-units-of-measure.org/resource/om-2/degreeCelsius
om:degreeCelsius
  rdf:type
    owl:NamedIndividual ,
    :TemperatureUnit .
\end{lstlisting}

The \lstinline|TemperatureMeasurement| class connects the different parts by referencing the sensor class, the property, its unit and the format, in which the measurement is made (\autoref{lst:temperaturemeasurement-class}).

\begin{lstlisting}[language=turtle,caption={{\ttfamily{}TemperatureMeasurement} class},label={lst:temperaturemeasurement-class}]
###  https://sdsc.fly.dev/ontology#TemperatureMeasurement
:TemperatureMeasurement
  rdf:type owl:Class ;
  rdfs:subClassOf
    saref:Measurement ,
    [ rdf:type owl:Restriction ;
      owl:onProperty saref:measurementMadeBy ;
      owl:someValuesFrom :TemperatureSensor
    ] ,
    [ rdf:type owl:Restriction ;
      owl:onProperty saref:relatesToProperty ;
      owl:someValuesFrom :Temperature
    ] ,
    [ rdf:type owl:Restriction ;
      owl:onProperty saref:isMeasuredIn ;
      owl:someValuesFrom :TemperatureUnit
    ] ,
    [ rdf:type owl:Restriction ;
      owl:onProperty saref:hasValue ;
      owl:allValuesFrom xsd:float
    ] .
\end{lstlisting}
