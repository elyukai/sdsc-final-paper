\section{Problem Statement}\label{sec:problem-statement}

The data published by cities does not have a uniform format, with both structured formats such as \gls{json} and unstructured formats such as plain text being used, which makes automated processing of the data more difficult \autocite{govdata}. Currently, there is no comprehensive solution to this problem, as existing systems only provide partial solutions or are not specific enough for the Smart City sensor application domain.

Determining whether the \gls{wot} and its associated \glspl{td} are a good solution is part of our problem, as so far it appears to be the best approach, providing structure and semantic expressiveness for this use case. However, the \gls{wot} does not make assumptions about the content of descriptions and ontologies used for Smart City applications are neither standardized nor provide a common vocabulary that covers a wide range of Smart City sensor application domains. Additionally, the \gls{wot} does not include \gls{qos} aspects, which can be important for \gls{sd}, enhancing the quality and preciseness of search.

In order to enable search, the \gls{wot} provides different search type options that may be implemented \autocite[section 7.3.2.3]{w3c.wot.discovery.20210602}. It is to be validated that at least of these search options is suitable for \gls{sd} of public sensors.

A challenge is to extend the \gls{wot} to solve all the mentioned problems. Additionally, it must also be tested in practice to check for its functionality. In \autoref{sec:pocapplication}, the implementation of a proof of concept is examined, and it is shown how this solution can be used in a test environment.
