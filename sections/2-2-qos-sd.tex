\subsection{Quality-of-Service-based \glsentrylong{sd}}\label{sec:qossd}

\glsdisp{qos}{Quality-of-Service-based (\glsxtrshort{qos})} \glspl{sd} are focused on non-functional evaluation criteria. The \gls{qos} aspect includes all measures over the quality of a communication service from the user's point of view (e.g. bandwidth, latency, transmission rate, \dots). \glsxtrshort{qos}-based \glspl{sd} can be divided into subcategories \autocite{Achir.2020.AtosdaiI}:

\begin{itemize}
\item Security-based
\item Energy-consumption-based
\item Network-infrastructure-based
\end{itemize}

Additionally, there are hybrid categories which can include a combination of two or all other categories.

The security-based aspect can be omitted for our work since we will be working with open data. Energy consumption and network infrastructure are considerably more important since most sensors have restricted resources.

In most cases, \glsxtrshort{qos}-based \glspl{sd} are based on other \gls{sd} approaches like semantic \gls{sd}. This makes them quite versatile. While the base \gls{sd} works just like it always does. After a service was found, the \glsxtrshort{qos}-based \gls{sd} acts as a kind of \gls{qos} filter \autocite{Kosunalp.2020.SArlbQaIsdm}. Thereby services that are not suitable get rejected.

One example of \glsxtrshort{qos}-based \gls{sd} is the concept of \emph{SARL}, which is based on a peer-to-peer concept that by itself does not cover \gls{qos} factors.
