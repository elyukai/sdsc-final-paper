\subsection{Backend}\label{sec:backend}

The \gls{wot} only provides the specification for a \gls{tdd} service \autocite[section 7.3]{w3c.wot.discovery.20210602}, but there are no full implementations of the standard yet. Therefore, we implemented the relevant parts of the Thing Description Directory specification.

A centralized service is necessary for identifying and accessing the \glspl{td} provided by multiple sources. The \gls{td} will be obtained through a single \gls{url} that leads to a provider service, which returns the \gls{td} metadata. The service caches the \gls{td} metadata in a database for efficient retrieval and provides an \gls{api} for accessing it. Additionally, the service keeps track of the respective providers for each \gls{td} and allows them to register and manage their own \glspl{td}, as well as store additional information about the providers themselves. Furthermore, the implementation of the service follows the \gls{tdd} guidelines specified by the \gls{wot}.

A \gls{sd} process starts with a request being sent to a thing provider, which returns the \gls{td} metadata. This is stored in a database along with other associated data such as the provider itself. A client can then retrieve one or more \glspl{td} via the \gls{api}.

\subsubsection{API}\label{sec:apibackend}

The \gls{api} is a trimmed down implementation of the \gls{wot} Directory Service \gls{api} and is responsible for retrieving, listing and searching \glspl{td}, while also offering an Events \gls{api} to notify clients about the changes to \glspl{td}. The ontology described earlier in this paper is provided over a separate endpoint.

The following list shows all openly accessible endpoints to the clients:

\begin{itemize}
    \item \lstinline|/ontology| shows our defined ontology.
    \item \lstinline|/api/things| returns a paginated list of \glspl{td}.
    \item \lstinline|/api/things/{id}| returns a single \gls{td}, where the \lstinline|id| parameter is the \gls{uri} of the \gls{td}.
    \item \lstinline|/api/search/jsonpath?query={query}| searches \glspl{td} with an \emph{JSONPath} expression as the query parameter.
    \item \lstinline|/events/{type}| informs about changes to \glspl{td} via \gls{SSE} where the type specifies the event.
\end{itemize}

\subsubsection{Admin-Panel}\label{sec:adminpanel}

The administration panel is a web app that manages the data of the \gls{tdd} as well as information about the organizations that provide \gls{td} metadata. Organizations administrators can register new \glspl{td} and edit their organization information.

